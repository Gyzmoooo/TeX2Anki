\documentclass{article}
\usepackage[utf8]{inputenc}
\usepackage[T1]{fontenc}
\usepackage{lmodern}
\usepackage[italian]{babel}
\usepackage{hyperref}
\usepackage{enumitem}
\usepackage{amsmath}
\usepackage{amssymb}

\begin{document}

\title{
  Precorso di Matematica \\
  \large Rielaborazione personale}
\author{Giulio Gismondi}
\date{Settembre 2025}
\maketitle

\tableofcontents


\section{Introduzione}
L'intento di questo documento è quello di realizzare una rielaborazione personale
del Precorso di Matematica dell'Università degli Studi di Bari Aldo Moro, tenuto nel
Settembre 2025 da Amedeo Altavilla e Gabriele Mancini. Tutti gli appunti originali 
del precorso sono presenti a questo \href{https://www.dm.uniba.it/it/didattica/precorsi/a-a-2025-26/precorso-di-matematica}{indirizzo}.


\section{Logica}
La logica è lo studio delle leggi del ragionamento e della dimostrazione.
Studiare la logica per un matematico è fondamentale. La logica trattata in questo documento
costituisce il linguaggio matematico di base: senza logica proposizionale e logica dei predicati, qualsiasi corso di matematica è fondamentalmente incomprensibile.
In questa rielaborazione ci saranno anche esempi in linguaggio naturale 
(nel nostro caso in lingua italiana): la logica ci aiuta anche ad analizzare e
trovare errori in quello che diciamo e che scriviamo.

Inoltre, studiare logica consente di osservare e comprendere i principi alla base
delle dimostrazioni e quindi della costruzione di oggetti e sistemi matematici.

\subsection{Logica proposizionale}
La logica, e in particolar modo quella proposizionale, tratta fondamentalmente di
proposizioni (anche dette enunciati). Una \textit{proposizione} è un'affermazione a cui è assegnabile
in modo chiaro e univoco un \textbf{valore di verità}; ovvero, un'affermazione che tutti riconosciamo 
essere sicuramente vera o sicuramente falsa. Questa affermazione può essere un'affermazione
ad argomento matematica o no. 

Ma facciamo degli esempi di affermazioni che sono o non sono proposizioni.
"Io mento", "Questa affermazione è falsa", "La logica è divertente" e "La logica non mi piace"
sono tutti esempi di affermazioni che non sono proposizioni. Le prime due affermazioni
portano a paradossi, mentre le ultime due esprimono giudizi soggettivi; per queste
ragioni, alle affermazioni elencate non è possibile assegnare un valore di verità.

"3 < 2", "$\sqrt{2}$ è irrazionale", "3 > 2" sono tutte proposizioni. A tutte queste
si può assegnare un valore di verità: la prima è falsa, la seconda e la terza sono vere.

Queste proposizioni possono essere legate fra loro e messe in relazione in moltissimi modi 
diversi. Per fare ciò si usano i \textit{connettivi}. Fra questi, tratteremo:
\begin{itemize}[itemsep=0.05mm]
  \item Negazione
  \item Congiunzione
  \item Disgiunzione
  \item Disgiunzione esclusiva
  \item Implicazione
  \item Doppia implicazione
\end{itemize} 

\subsubsection{Negazione}
La negazione è un connettivo \textit{unario} o \textit{monadico}, ovvero fa riferimento
a una sola proposizione.

Data una proposizione P, definiamo $\neg(P)$ la sua negazione. 
$\neg(P)$ è falso se P è vero, ed è vero se P è falso.

Facciamo allora la conoscenza dello strumento che più ci sarà utile per districarci
nel mondo delle proposizioni e dei connettivi. La \textit{tabella di verità} ci consente di 
distinguere tutti i possibili valori di verità di una possibile proposizione.
Tramite tabelle di verità possiamo definire connettivi e verificare l'equivalenza
logica fra due proposizioni. Questa, per esempio, è la tabella che definisce la negazione:

\vspace{0.3cm}
\begin{center}
  \begin{tabular}{|c|c|c|}
    \hline
    $P$ & $\neg(P)$ \\
    \hline
    V & F \\
    \hline
    F & V \\
    \hline
  \end{tabular}
\end{center}
Esempi di proposizioni con negazione: 
\begin{itemize}[itemsep=0.05mm]
  \item Se P è "Io ho finito di studiare", allora $\neg(P)$ è "Io \textbf{non} ho finito di studiare"
  \item Se Q è "2 < 3", allora $\neg(Q)$ è "$2 \geq 3$"
\end{itemize}
Una legge importante che riguarda la negazione è la \textbf{Legge della doppia negazione},
che afferma che 

\begin{center}
  \begin{align}
    \neg(\neg(P)) = P \label{1}
  \end{align}
\end{center}

\subsubsection{Congiunzione}
La \textit{congiunzione} è un connettivo \textit{binario}, ovvero viene utilizzato per
legare due proposizioni e costruirne una nuova. Il simbolo della congiunzione è $\land$. 
Prese due proposizioni A e B, la proposizione $A \land B$ è vera solo quando sono 
entrambe vere A e B. Il connettivo può essere definito con la seguente tabella di verità: 
\vspace{0.3cm}
\begin{center}
  \begin{tabular}{|c|c|c|}
    \hline
    $A$ & $B$ & $A \land B$ \\
    \hline
    V & V & V \\
    \hline
    V & F & F \\
    \hline
    F & V & F \\
    \hline
    F & F & F \\
    \hline
  \end{tabular}
\end{center}
\vspace{0.2cm}
Esempi di congiunzione: 
\begin{itemize}[itemsep=0.05mm]
  \item Se P è "Mi piace il blu" e Q è "Mi piace il rosso", allora $P \land Q$ è "Mi piace il blu \textbf{e} mi piace il rosso"
  \item Se R è "2 < 3" e S è "1 + 1 = 2" allora $R \land S$ è "2 < 3 \textbf{e} 1 + 1 = 2"
\end{itemize}

\subsubsection{Disgiunzione}
La \textit{disgiunzione} è un connettivo \textit{binario}, e il suo simbolo è $\lor$. 
Prese due proposizioni A e B, la proposizione $A \lor B$ è vera quando almeno
una fra A e B è vera:
\vspace{0.3cm}
\begin{center}
  \begin{tabular}{|c|c|c|}
    \hline
    $A$ & $B$ & $A \lor B$ \\
    \hline
    V & V & V \\
    \hline
    V & F & V \\
    \hline
    F & V & V \\
    \hline
    F & F & F \\
    \hline
  \end{tabular}
\end{center}
\vspace{0.2cm}
Esempi di disgiunzione: 
\begin{itemize}[itemsep=0.05mm]
  \item Se P è "Mi piace il viola" e Q è "Mi piace il verde", allora $P \land Q$ è "Mi piace il viola \textbf{o / oppure} mi piace il verde"
  \item Se R è "1 = 2" e S è "1 + 1 = 2" allora $R \land S$ è "1 = 2 \textbf{o / oppure} 1 + 1 = 2"
\end{itemize}
Notare come anche se è ovvio che R sia falso, la proposizione $R \lor S$ è vera, perchè
almeno una fra R e S è vera. 

\subsubsection{Disgiunzione esclusiva}
La \textit{disgiunzione eslcusiva} è un connettivo \textit{binario}, e il suo simbolo è $\dot{\lor}$ e si legge
"Aut". 
Prese due proposizioni A e B, la proposizione $A \dot{\lor} B$ è vera quando \textbf{esattamente una}
fra le due proposizioni è vera. $A \dot{\lor} B$ è definita dalla seguente tabella di verità:
\vspace{0.3cm}
\begin{center}
  \begin{tabular}{|c|c|c|}
    \hline
    $A$ & $B$ & $A \dot{\lor} B$ \\
    \hline
    V & V & F \\
    \hline
    V & F & V \\
    \hline
    F & V & V \\
    \hline
    F & F & F \\
    \hline
  \end{tabular}
\end{center}
\vspace{0.2cm}
Esempi di disgiunzione: 
\begin{itemize}[itemsep=0.05mm]
  \item Se P è "Mi piace il viola" e Q è "Mi piace il verde", allora $P \dot{\lor} Q$ è "\textbf{O} mi piace il viola \textbf{o} mi piace il verde"
  \item Se R è "1 = 2" e S è "1 + 1 = 2" allora  $R \dot{\lor} S$ è "\textbf{O} 1 = 2 \textbf{oppure} 1 + 1 = 2"
\end{itemize}

Nota bene: in realtà, la disgiunzione esclusiva può essere ricavata dai connettivi precedentemente
definiti. Infatti,
\vspace{-1.5em}
\begin{center}
  \begin{align}
    \neg((A \land B) \land (A \lor B)) \equiv A \dot{\lor} B \label{2}
  \end{align}
\end{center}
Dove $\equiv$ è il simbolo di equivalenza, che va a indicare che gli elementi 
$\neg((A \land B) \land (A \lor B))$ e $A \dot{\lor} B$ hanno la stessa tabella 
di verità.

\subsubsection{Leggi di De Morgan e proprietà dei connettivi}
Le proposizioni costruite con i connettivi rispettano delle leggi e godono di alcune proprietà. \\
Le leggi di De Morgan ci consentono di negare proposizioni che presentano congiunzioni
e disgiunzioni:

\begin{center}
  \vspace{-1.5em}
  \begin{align}
    \neg(A \land B) \equiv \neg(A) \lor \neg(B) \label{3} \\
    \neg(A \lor B) \equiv \neg(A) \land \neg(B) \label{4}
  \end{align}
\end{center}
Per quanto riguarda le proprietà delle proposizioni costruite con i connettivi, 
siano P, Q e R enunciati qualsiasi, valgono le seguenti equivalenze:
\begin{center}
  \vspace{-1.5em}
  \begin{align}
    P \land (Q \land R) \equiv (P \land Q) \land R \label{5}\\
    P \lor (Q \lor R) \equiv (P \lor Q) \lor R \label{6} \\
    P \land (Q \lor R) \equiv (P \land Q) \lor (P \land R) \label{7} \\
    P \lor (Q \land R) \equiv (P \lor Q) \land (P \lor R) \label{8}
  \end{align}
\end{center}

La \eqref{5} e la \eqref{6} rappresentano il corrispettivo della proprietà associativa,
mentre la \eqref{7} e la \eqref{8} corrispondono alla distributiva.

\subsubsection{Implicazione}




\subsubsection{Doppia implicazione}


\subsection{Logica dei predicati}
% Definizione di predicato

\subsubsection{Quantificatore universale}
\subsubsection{Quantificatore esistenziale}
% Qui va parlato anche di esiste ed è unico
\subsubsection{Negare i predicati}

\section{Teoria (ingenua) degli insiemi}
% Piccola spiegazione sul perchè viene considerata ingenua

\subsection{Termini primitivi}
% Descrivi caratteristiche degli insiemi: come possono essere descritti, ordine degli
% elementi, elementi compaiono solo una volta

\subsection{Sottoinsiemi e insieme vuoto}

\subsection{Insieme delle parti}

\subsection{Operazioni tra insiemi}
\subsubsection{Insieme unione}
\subsubsection{Insieme intersezione}
\subsubsection{Differenza insiemistica}
\subsubsection{Insieme complementare}
\subsubsection{Differenza simmetrica}
\subsubsection{Prodotto cartesiano}

\subsection{Proprietà degli insiemi e delle loro operazioni}
% Inserisci sia le proprietà che hai scritto subito dopo l'insieme complementare che quelle
% dopo la differenza simmetrica e il prodotto cartesiano

\section{Insiemi numerici}
Gli insiemi numerici sono particolari insiemi infiniti di numeri. I principali insiemi 
numerici sono:
\begin{itemize}[itemsep=0.05mm]
  \item Naturali, indicati con $\mathbb{N}$
  \item Interi (relativi), indicati con $\mathbb{Z}$
  \item Razionali, con $\mathbb{Q} = \{\frac{n}{m} | n \in  \mathbb{Z}, n \in \mathbb{Z} \setminus \{0\} \}$
  \item Reali, indicati con $\mathbb{R}$
  \item Irrazionali, indicati con $\mathbb{R} \setminus \mathbb{Q}$ 
  \item Complessi, indicati con $\mathbb{C} = \{x + iy | x, y \in \mathbb{R} i^2 = -1\}$ 
\end{itemize}
Per costruire matematicamente gli insiemi numerici ci sono tipicamente due approcci diversi.
Il primo approccio, quello più tradizionale, è di partire dalla definizione assiomatica 
di $\mathbb{N}$ fornita dagli \textit{Assiomi di Peano}. Il secondo invece, che è quello
che solitamente si segue nei corsi di Analisi 1, consiste nel costruire tutto a partire
 da  $\mathbb{R}$. In questo documento sono affrontati brevemente e sinteticamente
 entrambi i metodi. Per costruire con successo $\mathbb{N}$ e  $\mathbb{R}$ abbiamo
 tuttavia bisogno di una serie di concetti preliminari.

% Elenco insiemi numerici e breve descrizione
\subsection{Nozioni preliminari per la costruzione di insiemi numerici}
Abbiamo innanzitutto bisogno del concetto di funzione. 
\vspace{0.2cm}\\
\textbf{Definizione} \label{1_def} \\
Una funzione è una terna di oggetti (A, B, $f$) dove A e B sono insiemi non vuoti e 
$f$ è una relazione che ad ogni elemento di A associa uno e un solo elemento di B.
La terna può essere espressa come $f : A \to B$
\vspace{0.2cm} \\
\textbf{Definizione} \label{2_def}
Prendiamo due insiemi A e B e $f : A \to B$. $f$ è biettiva/bigiettiva se  
\vspace{-1.5em}
\begin{center}
  \begin{align}
    \forall y \in B \ \, \exists! x \in A \: t.c. \: f(x) = y \label{9}
  \end{align}
\end{center}
\textbf{Definizione} \label{3_def} \\
  Si definisce operazione binaria di un insieme A una qualsiasi funzione $f : A \times
   A \to A$

% Definizione di funzione, imagine, biettività, operazione binaria
\subsection{Definizione assiomatica di $\mathbb{N}$}
  A questo punto abbiamo tutto quel che ci serve per definire assiomaticamente $\mathbb{N}$.
  
  Prendiamo un insieme   $\mathbb{N}$, un elemento $\bar{n}$ di $\mathbb{N}$ detto \textit{
  Primo elemento} e una funzione $s : \mathbb{N} \to \mathbb{N}$ (dove s sta per successivo). \\
  L'insieme dei numeri Naturali è una terna ($\mathbb{N}$, $\bar{n}$, s) che soddisfa le
  seguenti proprietà:
  \begin{itemize}[itemsep=0.05mm]
    \item $\forall n, m \in  \mathbb{N}: (n \neq m) \implies (s(n) \neq s(m))$
    \item $s(\mathbb{N}) = \mathbb{N} \setminus \{\bar{n}\}$
    \item $(A \subseteq \mathbb{N} \: t.c. \: \bar{n} \in A \land s(A) \subseteq A) \implies A = \mathbb{N} $

\subsection{Definizione assiomatica di $\mathbb{R}$}
\subsubsection{Proprietà della somma}
\subsubsection{Proprietà del prodotto}
\subsubsection{Proprietà dell'ordinamento}
\subsubsection{Assioma di Dedekind}
\subsubsection{Conseguenze delle proprietà caratteristiche}

\subsection{Sottoinsiemi induttivi}
\subsection{Numeri primi e teorema fondamentale dell'aritmetica}
\subsection{$\mathbb{Z}$ e sue caratteristiche}
\subsubsection{Divisori e valore assoluto}
\subsubsection{Teorema di divisione euclidea e numero dei divisori    }
\subsubsection{MCD e mcm}
\subsubsection{Sommatoria e produttoria}

\subsection{$\mathbb{Q}$ e sue caratteristiche}
\subsubsection{Irrazionalità   di $\sqrt{2}$}
\subsubsection{Radice quadrata}

\end{document}